\documentclass{article}
\usepackage[utf8]{inputenc}
\usepackage[margin=2cm]{geometry}
\usepackage{amssymb}
\usepackage{booktabs}
\usepackage{geometry}
\usepackage{hyperref}
\usepackage[utf8]{inputenc}
\usepackage{mathtools}
\usepackage{MnSymbol}
\usepackage{pgfplots}
\usepackage{physics}
\usepackage{titling}

\pgfplotsset{compat=1.6}

\begin{document}
%--------------------------------------title--------------------------------------
\begin{flushright} STAT 2600 Recitation\\* 10/13/2020 \end{flushright}
\begin{center}
\textbf{Recitation Week 8 Problems}
\end{center}
%--------------------------------------body--------------------------------------
Today we will work on making some plots and drawing conclusions - we will also be exploring Markdown files. To begin, open a markdown file - this can be a .Rmd file (a markdown file in R studio), a google collab book, or a jupyter notebook. Formatting and presentation are important today.
 
    \begin{enumerate}

	\item[1) ]{Pick either a dataset we have used so far in recitation or a dataset pre-loaded in R that you find interesting.}
	
	\item[2) ]{Data diagnostics: make a histogram of at least one variable. What does this information tell you? Draw conclusions - type up your conclusions in a text cell next to your plot.}
	
	\item[3) ]{Data exploration: make at least one bivariate plot. What does this information tell you? Draw conclusions - type up your conclusions in a text cell below near you plot.}
   
     \end{enumerate}
\end{document}