\documentclass{article}
\usepackage[utf8]{inputenc}
\usepackage[margin=2cm]{geometry}
\usepackage{amssymb}
\usepackage{booktabs}
\usepackage{geometry}
\usepackage{hyperref}
\usepackage[utf8]{inputenc}
\usepackage{mathtools}
\usepackage{MnSymbol}
\usepackage{pgfplots}
\usepackage{physics}
\usepackage{titling}

\pgfplotsset{compat=1.6}

\begin{document}
%--------------------------------------title--------------------------------------
\begin{flushright} STAT 2600 Recitation\\* 9/29/2020 \end{flushright}
\begin{center}
\textbf{Recitation Week 6 Problems}
\end{center}
%--------------------------------------body--------------------------------------
Today we will work with the ```iris``` dataset built into R.
 
    \begin{enumerate}

	\item[1) ]{How many unique species of flowers are there in this dataset?.}
	
	\item[2) ]{(Is this data long or wide? If it is wide convert it to long, if it is long convert it to wide.}
	
	\item[3) ]{What is the frequency of each flower species? What are the values of Sepal.Length? Does it make sense to calculate the frequencies of this variable? If not how else could you look at this information?}
   
     \end{enumerate}
\end{document}