\documentclass{article}
\usepackage[utf8]{inputenc}
\usepackage[margin=2cm]{geometry}
\usepackage{amssymb}
\usepackage{booktabs}
\usepackage{geometry}
\usepackage{hyperref}
\usepackage[utf8]{inputenc}
\usepackage{mathtools}
\usepackage{MnSymbol}
\usepackage{pgfplots}
\usepackage{physics}
\usepackage{titling}

\pgfplotsset{compat=1.6}

\begin{document}
%--------------------------------------title--------------------------------------
\begin{flushright} STAT 2600 Recitation\\* 12/1/2020 \end{flushright}
\begin{center}
\textbf{Recitation Week 15 Problems}
\end{center}
%--------------------------------------body--------------------------------------
Pick any dataset you are familiar with (except for the Iris and Diamonds data) and build a simple KNN model. This should include the following steps (code for these can be found in the notes from class):
 
    \begin{enumerate}

	\item[1) ]{Pick an outcome variable (what do you want to predict?) and a set of characteristics you want to predict based on.}
	
	\item[2) ]{Split your dataset into test and train. Standardize your variables. Bonus: why is it important to standardize variables?}
	
	\item[3) ]{Run the KNN and examine your predictions. Make a plot of your predictions - give a geometric interpretation of KNN algorithm.}
	
	\item[4) ]{Examine / interpret the confusion matrix, and calculate the accuracy.}
   
     \end{enumerate}
\end{document}


