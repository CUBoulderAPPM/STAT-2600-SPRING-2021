\documentclass{article}
\usepackage[utf8]{inputenc}
\usepackage[margin=2cm]{geometry}
\usepackage{amssymb}
\usepackage{booktabs}
\usepackage{geometry}
\usepackage{hyperref}
\usepackage[utf8]{inputenc}
\usepackage{mathtools}
\usepackage{MnSymbol}
\usepackage{pgfplots}
\usepackage{physics}
\usepackage{titling}

\pgfplotsset{compat=1.6}

\begin{document}
%--------------------------------------title--------------------------------------
\begin{flushright} STAT 2600 Recitation\\* 10/6/2020 \end{flushright}
\begin{center}
\textbf{Recitation Week 7 Problems}
\end{center}
%--------------------------------------body--------------------------------------
Today we will work with some COVID-19 time series data from Our World in Data to practice basic data diagnostics and plotting with ggplot. Go download the covid 19 time series data here: https://ourworldindata.org/coronavirus-source-data. Save to your computer and import into R.
 
    \begin{enumerate}

	\item[1) ]{What are data diagnostics? Why is important to diagnose issues in your data before you plot?}
	
	\item[2) ]{Is this data wide or long?}
	
	\item[3) ]{What are you curious about in this data? What plots would you make to investigate your questions?}
	
	\item[4) ]{Make a line plot of total cases by day - have one line for the US, one for South Korea, and one for all of Europe (hint: you may need to make a lookup for european countries to do this part)}
   
     \end{enumerate}
\end{document}